%% This is file `elsarticle-template-1-num.tex',
%%
%% Copyright 2009 Elsevier Ltd
%%
%% This file is part of the 'Elsarticle Bundle'.
%% ---------------------------------------------
%%
%% It may be distributed under the conditions of the LaTeX Project Public
%% License, either version 1.2 of this license or (at your option) any
%% later version.  The latest version of this license is in
%%    http://www.latex-project.org/lppl.txt
%% and version 1.2 or later is part of all distributions of LaTeX
%% version 1999/12/01 or later.
%%
%% The list of all files belonging to the 'Elsarticle Bundle' is
%% given in the file `manifest.txt'.
%%
%% Template article for Elsevier's document class `elsarticle'
%% with numbered style bibliographic references
%%
%% $Id: elsarticle-template-1-num.tex 149 2009-10-08 05:01:15Z rishi $
%% $URL: http://lenova.river-valley.com/svn/elsbst/trunk/elsarticle-template-1-num.tex $
%%
\documentclass[preprint,12pt]{elsarticle}

%% Use the option review to obtain double line spacing
%% \documentclass[preprint,review,12pt]{elsarticle}

%% Use the options 1p,twocolumn; 3p; 3p,twocolumn; 5p; or 5p,twocolumn
%% for a journal layout:
%% \documentclass[final,1p,times]{elsarticle}
%% \documentclass[final,1p,times,twocolumn]{elsarticle}
%% \documentclass[final,3p,times]{elsarticle}
%% \documentclass[final,3p,times,twocolumn]{elsarticle}
%% \documentclass[final,5p,times]{elsarticle}
%% \documentclass[final,5p,times,twocolumn]{elsarticle}

\usepackage[utf8]{inputenc}

%% if you use PostScript figures in your article
%% use the graphics package for simple commands
%% \usepackage{graphics}
%% or use the graphicx package for more complicated commands
%% \usepackage{graphicx}
%% or use the epsfig package if you prefer to use the old commands
%% \usepackage{epsfig}

%% The amssymb package provides various useful mathematical symbols
\usepackage{amssymb}
%% The amsthm package provides extended theorem environments
%% \usepackage{amsthm}

%% The lineno packages adds line numbers. Start line numbering with
%% \begin{linenumbers}, end it with \end{linenumbers}. Or switch it on
%% for the whole article with \linenumbers after \end{frontmatter}.
\usepackage{lineno}

%% natbib.sty is loaded by default. However, natbib options can be
%% provided with \biboptions{...} command. Following options are
%% valid:

%%   round  -  round parentheses are used (default)
%%   square -  square brackets are used   [option]
%%   curly  -  curly braces are used      {option}
%%   angle  -  angle brackets are used    <option>
%%   semicolon  -  multiple citations separated by semi-colon
%%   colon  - same as semicolon, an earlier confusion
%%   comma  -  separated by comma
%%   numbers-  selects numerical citations
%%   super  -  numerical citations as superscripts
%%   sort   -  sorts multiple citations according to order in ref. list
%%   sort&compress   -  like sort, but also compresses numerical citations
%%   compress - compresses without sorting
%%
%% \biboptions{comma,round}

% \biboptions{}


\journal{Documento Interno}

\begin{document}

\begin{frontmatter}

%% Title, authors and addresses

%% use the tnoteref command within \title for footnotes;
%% use the tnotetext command for the associated footnote;
%% use the fnref command within \author or \address for footnotes;
%% use the fntext command for the associated footnote;
%% use the corref command within \author for corresponding author footnotes;
%% use the cortext command for the associated footnote;
%% use the ead command for the email address,
%% and the form \ead[url] for the home page:
%%
%% \title{Title\tnoteref{label1}}
%% \tnotetext[label1]{}
%% \author{Name\corref{cor1}\fnref{label2}}
%% \ead{email address}
%% \ead[url]{home page}
%% \fntext[label2]{}
%% \cortext[cor1]{}
%% \address{Address\fnref{label3}}
%% \fntext[label3]{}

\title{Definición del problema distribuido de la flota de vehículos}

%% use optional labels to link authors explicitly to addresses:
\author[label1]{J.I. Estévez-Damas}
\address[label1]{Universidad de La Laguna, Spain}
%% \address[label2]{<address>}

\begin{abstract}
%% Text of abstract
En este documento voy a plantear un problema del tipo ``dial a ring'' que requiere una implementación distribuida. Se elimina la red global de comunicaciones de modo que cada robot sólo puede comunicar con los robots y otros elementos de comunicación que están en su zona de cobertura. Las peticiones de servicio son establecidas por los usuarios en dispositivos de comunicación que hacen periódicamente envíos hasta que un robot de transporte u otro tipo de robot de comunicación las recibe. Las peticiones pendientes pueden ser enviadas a otros robots que entran en la zona del robot. La información sobre peticiones en curso de ser atendidas también se difunde de robot a robot. 

\end{abstract}

\begin{keyword}
Autonomous Vehicle, Robotic Fleet, Distributed Algorithms
%% keywords here, in the form: keyword \sep keyword

%% MSC codes here, in the form: \MSC code \sep code
%% or \MSC[2008] code \sep code (2000 is the default)

\end{keyword}

\end{frontmatter}

%%
%% Start line numbering here if you want
%%
\linenumbers

%% main text
\section{Introducción}

Nos planteamos un escenario sin red de comunicaciones global para los robots de la flota, ni elemento central de cómputo capaz de resolver el problema global. Por el contrario, cada robot tendrá los recursos de cómputo necesarios para resolver el problema de optimización dial-a-ring con un depot. Veamos en más detalle las reglas de este problema:

\begin{itemize}

\item Los usarios están en localizaciones conectadas por rutas. Una \textbf{petición de servicio} es activada por el usuario mediante un elemento de comunicación, que reenvía constantemente la solicitud hasta que un robot asume la petición. Para ello, el robot y el elemento de comunicación del usuario deben \textbf{intersectar sus zonas de cobertura}. El robot lo incorpora a la \textbf{lista de peticiones pendientes local} y el elemento de comunicación deja de transmitir.

\item Cuando un robot intersecta su cobertura con otro, dan comienzo a un proceso de \textbf{fusión de listas} (ver más abajo), afectando a la \textbf{lista de peticiones pendientes} y \textbf{lista de peticiones atendidas}.

\item Las peticiones pendientes y peticiones atendidas se mantienen en buffers de tamaño finito. Cuando una nueva información es adquirida por el robot, la información más antigua en los buffers es descartada.

\item{Cuando un robot en \textbf{modo servicio} termina de atender una petición en curso llegando al \textbf{punto de destino} pasa a modo \textbf{coordinador-master}. En este modo cada cierto tiempo el robot tratará de realizar un \textbf{proceso de coordinación}.}

\item{Un proceso de coordinación consiste en:
\begin{itemize}
\item{Búsqueda de robots en \textbf{modo espera}. Cada vez que encuentra uno lo sitúa en \textbf{modo coordinador-esclavo}. }
\item{Finalizado el proceso de búsqueda, se procede a fusionar listas de peticiones pendientes y listas de peticiones atendidas entre los robots participantes en el modo coordinador iniciado.}
\item{El robot coordinador-máster resuelve un problema de asignación de rutas a partir de la lista de peticiones pendientes resultante de la fusión de listas.}
\item{Los robots con ruta asignada pasan a \textbf{modo servicio} y van a atender la ruta.}
\item{Los robots sin ruta asignada pasan a \textbf{modo espera}.}
\end{itemize}
}

\item{La fusión de listas consiste en:

\item Realizar la unión de listas de peticiones atendidas, esta será la nueva lista de peticiones atendidas en todos los robots.
\item Realizar la unión de listas de peticiones pendientes. Eliminar de esta lista todas aquellas peticiones que estén en la nueva lista de peticiones atendidas.

} 

\end{itemize}

\section{Formalización}

En cualquier instante tenemos:

\begin{itemize}

\item Un conjunto de vehículos autónomos: $V=\{v_1,v_2,\dots,v_{N_V}\}$ 
\item Un conjunto de localizaciones: $P=\{p_1, p_2, \dots, p_{N_p}\}$
\item Un conjunto de rutas: $R=\{r_1, r_2, \dots, r_{N_r}$ donde $r_i:P \rightarrow \left(P \cup \{ \bot \}\right)$.
\item El conjunto de pares de lugares conectados $CP=\{(p_i,p_j) | p_i, p_j \in P  \wedge \exists r_k \in R : r_k(p_i)=p_j \}$
\item La función de coste: $F_C:P \times P \times R \rightarrow \Re^{+}$. La función de coste cumple que $\forall p_i,p_j \in P, r_k \in R:  r_k(p_i) \neq p_j$ entonces $F_C(p_i,p_j,r_k)=\infty$.
\item Dados dos lugares, la función ruta óptima se define $r_{opt}:CP \rightarrow 2^R$ como $r_k \in r_{opt}(p_i,p_j)$ si $r_k(p_i)=p_j$ y $ \forall r_l \neq r_k: r_l(p_i)=p_j, F_C(p_i,p_j,r_k) \leq F_C(p_i,p_j,r_l)$.
\item La función de coste es monótona si  $\forall p_i, p_j, p_k \in P, r_p, r_q, r_r \in R$, si $r_p \in r_{opt}(p_i,p_j)  \wedge r_q \in r_{opt}(p_i,p_k) \wedge r_r \in r_{opt}(p_k,p_j)$ se cumple que $F_C(p_i,pj,r_p) \geq F_C(p_i,p_k,r_q)+F_C(p_k,p_j,r_r)$

\end{itemize}



\section{Análisis}

\bibliographystyle{elsarticle-num-names}
\bibliography{sample.bib}

%% Authors are advised to submit their bibtex database files. They are
%% requested to list a bibtex style file in the manuscript if they do
%% not want to use model1-num-names.bst.

%% References without bibTeX database:

% \begin{thebibliography}{00}

%% \bibitem must have the following form:
%%   \bibitem{key}...
%%

% \bibitem{}

% \end{thebibliography}


\end{document}

%%
%% End of file `elsarticle-template-1-num.tex'.
